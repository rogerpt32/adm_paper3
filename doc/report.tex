\documentclass[conference]{IEEEtran}
\usepackage[utf8]{inputenc}
\usepackage{booktabs}
\usepackage{enumitem} %resume enumerate
\usepackage{graphicx}
\usepackage{caption}
\usepackage{subcaption}
\usepackage{url}

\def\BibTeX{{\rm B\kern-.05em{\sc i\kern-.025em b}\kern-.08em
    T\kern-.1667em\lower.7ex\hbox{E}\kern-.125emX}}

\begin{document}

\title{Human Activity Recognition Using Smartphones}

\author{\IEEEauthorblockN{Roger Pujol}
\IEEEauthorblockA{\textit{Universitat Politècnica de Catalunya (UPC)}\\
Barcelona, Spain \\
roger.pujol.torramorell@est.fib.upc.edu}}

\date{\today}

\maketitle

\begin{abstract}
Nowadays everybody carries a smartphone and these devices are full of sensors. This includes accelerometers which could be used to get information about the activity of the person carrying it. In this paper we will use the information given by the sensors of a smartphone to recognize the activity done by the owner. Since this is a classification problem, we will use and analyse different methods in order to see which is the better for this particular case.\\
The code of this project is Open Source and can be found in: \url{https://github.com/rogerpt32/adm_paper3}
\end{abstract}

\section{Introduction}

% \bibliographystyle{unsrt}
% \bibliography{cites}

\end{document}